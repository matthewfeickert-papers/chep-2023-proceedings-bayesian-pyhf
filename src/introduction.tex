\section{Introduction}\label{sec:introduction}

The evaluation of High Energy Physics measurements depend on the comparison with theoretical predictions. The phenomenology of the observation is represented by a statistical model $p(x | \theta)$, i.e. the probability distribution of data $x$ for specific theory parameters $\theta$.
Given actual observations $x$, the likelihood $\mathcal{L}(\theta)$ can then be interpreted as $p(x | \theta)$ with fixed $x$ and is a measure of the compatibility between the observed data $x$ and the theory prediction depending on $\theta$.   \\ \\
Based on $p(x | \theta)$ there are different approaches to evaluating the model. In the frequentist setting inference methodologies include maximum-likelihood point estimation, hypothesis tests and confidence interval estimation. Applying Bayesian statistics to $p(x | \theta)$ is a different approach where the likelihood is used to update a prior belief $p(\theta)$ to a posterior belief $p(\theta|x)$ about the probable values of the model parameters $\theta$. \\
In particle physics statistical models are often represented based on templates such as \texttt{HistFactory}. \texttt{HistFactory} is a mathematical framework for building statistical models of binned analyses across different channels, see Sec.~\ref{subsec:HFandpyhf}. \texttt{RooFit} \cite{root} is a framework that already allows for Bayesian inference for \texttt{HistFactory} models, its range of application though is limited by the lack of the implementation of gradients and availability of advanced diagnostics for Bayesian inference results due to the historical focus on frequentist inference in HEP. An example for a library that allows for advanced Bayesian inference for particle and astro-physics is \texttt{BAT.jl} \cite{Schulz:2021BAT} but tools to construct \texttt{HistFactory} models within Julia are not yet readily available~\footnote{the LiteHF.jl~\cite{LiteHF} package is working towards HistFactory within a Julia context }. \\
\texttt{pyhf} is a Python library that implements the \texttt{HistFactory} template and already allows for frequentist inference \cite{pyhf, pyhf_joss}. It is the aim of this work to utilize the Python library \texttt{PyMC} and the automatic differentiation capabilities of \texttt{pyhf} to enable advanced Bayesian analysis for \texttt{HistFactory} models.
